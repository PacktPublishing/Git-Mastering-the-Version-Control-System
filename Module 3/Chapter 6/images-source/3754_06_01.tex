\documentclass[tikz]{standalone}

\usepackage[T1]{fontenc}
\usepackage[utf8]{inputenc}
\usepackage{times}

\usepackage{tikz}
\usetikzlibrary{matrix,graphs,arrows,shadings,positioning}
\usetikzlibrary{shapes.geometric,shapes.arrows,shapes.misc}
\usetikzlibrary{backgrounds,fit,shadows}


\begin{document}

\begin{tikzpicture}[>=stealth',thick,draw]
	\input{3754_styles.tex}
	\begin{scope}[yshift=+1cm]
	\matrix {
		&
		  % c1
		|[branch](maint)| maint
		& % c2
		& % c3
		& % c4
		|[branch](master)| master
		& % c5
		& % c6
		|[branch](next)| next \\
		%
		|[reflog,opacity=0](hspace)| unstable &
		|(c1)| C1 &
		|(c2)| C2 &
		|(c3)| C3 &
		|(c4)| C4 &
		|(c5)| C5 &
		|(c6)| C6 &
		|[fake,opacity=0](hspace)| master \\
	};
	\graph [use existing nodes] {
	    c1 <- c2 <- c3 <- c4 <- c5 <- c6;
	    c1 <- maint;
	    c4 <- master;
	    c6 <- next;
	};
	\end{scope}
	\begin{scope}[yshift=-3.1cm,
		every matrix/.append style={row sep=1.0cm},
		multi reflog/.style={
			reflog,
			align=center,
			text width=1cm,
			text depth=,
			anchor=center,
			yshift=0.2cm,
		},
		ll/.style={
			opacity=0.5,
			>=latex,
		}]
	\matrix {
		|[multi reflog](maint)| very stable &
		|(c1)| C1 &
		& % c2
		& % c3
		& % c4
		& % c5
		& % c6
		|[fake](x2)| maint \\
		%
		|[reflog](master)| stable &
		& % c1
		|(c2)| C2 &
		|(c3)| C3 &
		|(c4)| C4 &
		& % c5
		& % c6
		|[fake](x5)| master \\
		%
		|[reflog](next)| unstable &
		& % c1
		& % c2
		& % c3
		& % c4
		|(c5)| C5 &
		|(c6)| C6 &
		|[fake](x6)| next \\
	};
	\graph [use existing nodes] {
		c1 <- c2 <- c3 <- c4 <- c5 <- c6;
		maint ->[ll] master ->[ll] next;
	};
	\begin{scope} [on background layer,%
		bkg/.style={
			fill=purple!35,
			draw=none,
		},
		every node/.style={
			single arrow,
			single arrow tip angle=60,
			bkg,
		}]
		\node [fit=(c1)(x2)] (fmaint) {};
		\node [fit=(c2)(x5)] (fmaster) {};
		\node [fit=(c5)(x6)] (fnext) {};
		%
		\fill [bkg] % was: (fmaster.before tail)
			(c1.south west) -- (fmaster.before tail) --
			(c2.east) -- (c1.east) -- cycle;
		\fill [bkg]
			(c4.south west) -- (fnext.before tail) --
			(c5.east) -- (c4.east) -- cycle;
	\end{scope} % [on background layer]
	\end{scope}
\end{tikzpicture}
\end{document}
